
The present document is an official submission to the 13$^{th}$
contest of the ForMet sub-governance: {\em 13 Radiance-DEX Phase 0
  Formal Verification}
\url{https://formet.gov.freeton.org/proposal?isGlobal=false&proposalAddress=0%3A07783c48e8789fa1163699e9e3071a47917284724c1ff0d0e54f0d40beb86c07}

  The specification was: {\em The contestants shall provide the
    informal audit of the central Radiance-DEX smart contracts ({\tt
      DEXClient}, {\tt DEXConnector}, {\tt DEXPair}, {\tt DEXRoot
      RootTokenContract}, {\tt TONTokenWallet}), hereinafter referred
    to as ``smart contracts''.  where the detailed description of the
    ``informal audit'' is described below. All debot contracts are
    excluded from the present contest.  }

  and {\em All the source codes must be provided by the authors via
    \url{https://t.me/joinchat/-3zDgM62gQ02OGUy} Telegram group. The code to
    be audited has a hash 7d65f0d3b85e504ac33f01395b6ba0ffef9d5fe5
    (branch main, link -
    \url{https://github.com/radianceteam/dex2/commit/7d65f0d3b85e504ac33f01395b6ba0ffef9d5fe5})
  }

  and finally {\em
Contestants shall submit a document in PDF format that covers:
\begin{itemize}
  \item All the errors found
  \item All the warnings found
  \item All the ``bad code'' (long functions, violation of abstraction levels, poor readability etc.)
\end{itemize}

Errors and warnings should be submitted to the developers as early as
possible, during the contest, so that the code can be fixed
immediately.  }

During this audit, we classified our findings into three kinds of issues:
\begin{description}
\item[Critical Issues:] such issues can lead to taking ownership of resources,
or total disabling of the service;
\item[Major Issues:] such issues can lead to a decrease in the quality of the
service, or temporary loss of availability;
\item[Minor Issues:] Such issues do not impact the service itself. For example,
code improvements to improve readability, to improve sharing, etc.
\end{description}

In these three kinds of issues, we found in the DEX contracts:
\begin{description}

\item[{\bf 4 Critical issues:}] one is a data structure with unlimited growth; one is an attacker taking ownership of the liquidity root contract; one is the possibility for an attacker to impersonate another user to create a client contract with its balance; one is the failure of refunds for unsuccessful swaps;

\item[Many Major issues:] all of them are related to the use of {\tt tvm.accept}, allowing attackers to drain the balance of contracts. However, as the contracts are not expected to keep large amounts of TONs, we classified these attacks in the ``Major issues'' kind, as their impact will only be a loss of quality of service;

\item[Minor issues:] most of them are readability improvements;
\end{description}

Critical issues and major issues are listed in the ``Table of Critical
and Major Issues'' before this chapter. {\bf All critical issues and
most major issues have been submitted to the Radiance team developers}
on the Telegram channel above.

In the Broxus token contracts, we found one issue that we consider a
Critical issue: the constructor allows anybody to deploy a {\tt
RootTokenContract} with any static variables. It could be used by
attackers to preventively take control over some root contracts before
they are actually needed.  We also found many methods where a {\tt
require} should check that there is enough gas to send further
messages, leading to potential burns of tokens. We chose to define
these issues as Major issues.
