
In this chapter, we introduce some general remarks about the code,
that are not specific to a specific piece of code, but whose
occurrences have been found in the project in several locations.

\section{Readability}

\subsection{Typography of Static Variables}
\label{readability:static}

A good coding convention is to use typography to visually discriminate
static variables from other variables, for example using a prefix such
as {\tt s\_}.

This issue was found everywhere in the code of DEX and token
contracts.\\

\subsection{Typography of Global Variables}
\label{readability:global}

A good coding convention is to use typography to visually discriminate
global variables from local variables, for example using a prefix such
as {\tt m\_} or {\tt g\_}.

This issue was found everywhere in the code of DEX and token contracts.\\

\subsection{Typography of Internal Functions}
\label{readability:internal}

A good coding convention is to use typography to visually discriminate
public functions and internal functions, for example using a prefix
such as {\tt \_}.

This issue was found everywhere in the code of DEX and token contracts.\\

\subsection{Naming of Numbers}
\label{readability:constants}

A good coding convention is to define constants instead of using
direct numbers for errors and other meaningful numbers.

This issue was found everywhere in the code of DEX contracts (for
errors in {\tt require()} and payload opcodes), but not for token
contracts.

\subsection{Better Units for Big Numbers}
\label{readability:numbers}

A good coding convention is to use decimals of {\tt ton} instead of
default nanotons to decrease the size of integer constants.

This issue was found in all constant definitions for gas cost.
Numbers like {\tt 500000000} are difficult to read, whereas the
equivalent {\tt 0.5 ton} is much easier.

\subsection{Use Method Calls instead of {\tt tvm.encodeBody}}
\label{readability:call}

Using {\tt tvm.encodeBody} makes code harder to read. Method calls are
easier to read and interpret. The issue is minor as checks are still
correctly performed on argument types.

This issue was found in all DEX contracts except DEXRoot.

\section{Gas Consumption}

\subsection{Accept Methods without Checks}
\label{accept:all}

Public methods using {\tt tvm.accept()} without any prior check should
not exist. Indeed, such methods could be used by attackers to drain
the balance of the contracts, even with minor amounts but unlimited
number of messages.

This issue was found in the code the DEX contracts, especially with
the {\tt alwaysAccept()} modifier. Methods using this modifier should
check the origin of the message and limit {\tt tvm.accept()} to either
the user or known contracts.

\subsection{{\tt require} after {\tt tvm.accept}}
\label{accept:require}

Methods using {\tt tvm.accept()} should never use {\tt require()}
after the accept.  Indeed, a {\tt require()} failing after {\tt
  tvm.accept()} will cost a huge amount of gas, as all shards will
execute the failing method.

This issue was found in the code of the DEX contracts. Methods should
always keep calls to {\tt require()} before {\tt tvm.accept()}, and if
it is not possible, should not fail but should {\em return} an error
code instead.

\subsection{Not Enough Gas for Action}
\label{gas:partial}

If there is not enough gas (message value or balance), the compute
phase may execute, but the action phase may fail. In such a case,
modifications done during the compute phase are committed to the
blockchain, but no message emitted.

This issue was found in the code, for example in {\tt
  DEXClient.connectRoot}.

\section{Architecture}


\subsection{No need for passing {\tt souint} Arguments around}

It looks like there is little need for passing around the {\tt souint}
arguments as the same {\tt soUINT} static value could be used for the
{\tt DEXroot} and all other contracts derived from it. The only
required modification would be to make the {\tt drivenRoot} field of
{\tt DEXConnector} a static variable. Such a change would probably
simplify the interface of many functions.

