
In this chapter, we introduce some general remarks about the code,
that are not specific to this project, but whose occurrences have been
found in the project in several locations.

\section{Readability}

\subsection{Typography}

\issueMinor{Typography of static variables}{A good coding convention
  is to use typography to visually discriminate static variables from
  other variables, for example using a prefix such as {\tt s\_}. }

This issue was found everywhere in the code of DEX and token
contracts.\\

\issueMinor{Typography of global variables}{A good coding convention
  is to use typography to visually discriminate global variables from
  local variables, for example using a prefix such as {\tt m\_} or
  {\tt g\_}. }

This issue was found everywhere in the code of DEX and token contracts.\\

\issueMinor{Typography of internal function}{A good coding convention
  is to use typography to visually discriminate public functions and
  internal functions, for example using a prefix such as {\tt \_}. }

This issue was found everywhere in the code of DEX and token contracts.\\

\subsection{Naming of Errors}

\issueMinor{Naming of Errors}{A good coding convention is to define
  constants instead of using direct numbers for errors in {\tt
    require()}.}

This issue was found everywhere in the code of DEX contracts, but not
for token contracts.

\section{Security}

\issueCritical{No Accept All Methods}{Public methods using
  {\tt tvm.accept()} without any prior check should not exist. Indeed,
  such methods could be used by attackers to drain the balance of the
  contracts, even with minor amounts but unlimited number of
  messages.}

This issue was found in the code the DEX contracts, especially with
the {\tt alwaysAccept()} modifier. Methods using this modifier should
check the origin of the message and limit {\tt tvm.accept()} to either
the user or known contracts.

\issueCritical{No Require after Accept}{Methods using {\tt
    tvm.accept()} should never use {\tt require()} after the accept.
  Indeed, a {\tt require()} failing after {\tt tvm.accept()} will cost
  a huge amount of gas, as all shards will execute the failing
  method. }

This issue was found in the code of the DEX contracts. Methods should
always keep calls to {\tt require()} before {\tt tvm.accept()}, and if
it is not possible, should not fail but should {\em return} an error
code instead.

