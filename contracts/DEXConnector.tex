
\chapter{Contract DEXConnector}

\minitoc

In file {\tt DEXConnector.sol}

\section{Contract Inheritance}


\noindent\begin{tabular}{|l|p{5cm}|}\hline
IExpectedWalletAddressCallback & \\\hline
IDEXConnector & \\\hline
\end{tabular}


\section{Constant Definitions}

\begin{itemize}
\item \issueUnits{}
\end{itemize}

\begin{lstlisting}[firstnumber=19]
  uint128 constant GRAMS_TO_ROOT = 500000000;
\end{lstlisting}

\begin{lstlisting}[firstnumber=20]
  uint128 constant GRAMS_TO_NEW_WALLET = 250000000;
\end{lstlisting}

\section{Static Variable Definitions}

\begin{itemize}
\item \issueStatic{}
\item Minor issue: why is {\tt drivenRoot} not a static variable ? it
  looks like there is only one possible {\tt DEXConnector} for a given
  pair of {\tt DEXClient} and {\tt RootTokenContract}. Using {\tt
    drivenRoot} as a static would also make the need to pass {\tt
    souint} around useless, since the same {\tt souint} could be used
  everywhere, from the {\tt DEXRoot} to all the clients, pairs and
  connectors created from it.
\item Minor issue: the name {\tt dexclient} is misleading. In {\tt
  DEXPair.connectRoot}, connectors are created for DEXPair passing
  {\tt this} as {\tt dexclient}, i.e. an address of type {\tt
    DEXPair}. Maybe use {\tt owner\_address} ?
\end{itemize}

\begin{lstlisting}[firstnumber=15]
  uint256 static public soUINT;
\end{lstlisting}

\begin{lstlisting}[firstnumber=16]
  address static public dexclient;
\end{lstlisting}

\section{Variable Definitions}

\begin{itemize}
\item \issueGlobal{}
\end{itemize}

\begin{lstlisting}[firstnumber=22]
  address public drivenRoot;
\end{lstlisting}

\begin{lstlisting}[firstnumber=23]
  address public driven;
\end{lstlisting}

\begin{lstlisting}[firstnumber=24]
  bool public statusConnected;
\end{lstlisting}

\section{Modifier Definitions}

\subsection{Modifier alwaysAccept}

\begin{itemize}
\item \issueAlwaysAccept{DEXConnector}
\end{itemize}

\begin{lstlisting}[firstnumber=27]
	modifier alwaysAccept {
		tvm.accept();
		_;
	}
\end{lstlisting}

\subsection{Modifier checkOwnerAndAccept}

\begin{itemize}
\item \issueNumber{}
\end{itemize}

\begin{lstlisting}[firstnumber=32]
  modifier checkOwnerAndAccept {
    // Check that message from contract owner.
    require(msg.sender == dexclient, 101);
    tvm.accept();
    _;
  }
\end{lstlisting}

\section{Constructor Definitions}

\subsection{Constructor}

\begin{itemize}
\item OK
\end{itemize}

\begin{lstlisting}[firstnumber=39]
  constructor() public checkOwnerAndAccept {
      statusConnected = false;
  }
\end{lstlisting}

\section{Public Method Definitions}

\subsection{Receive function}

\begin{itemize}
\item OK
\end{itemize}

\begin{lstlisting}[firstnumber=129]
  receive() external {
  }
\end{lstlisting}

\subsection{Function burn}

\begin{itemize}
\item Minor Issue: this method should check {\tt require(statusConnected,..)}
\item \issueError{}
\item \issueEncodeBody{}
\end{itemize}

\begin{lstlisting}[firstnumber=116]
  function burn(uint128 tokens, address callback_address, TvmCell callback_payload) public override {
    require(msg.sender == dexclient, 101);
    tvm.rawReserve(address(this).balance - msg.value, 2);
    TvmCell body = tvm.encodeBody(IBurnableByOwnerTokenWallet(driven).burnByOwner, tokens, 0, dexclient, callback_address, callback_payload);
    driven.transfer({value: 0, bounce:true, flag: 128, body:body});
  }
\end{lstlisting}

\subsection{Function deployEmptyWallet}

\begin{itemize}
\item Minor Issue: if this method is called twice with different
  roots, the second one may still be executed before {\tt
    statusConnected} is set in {\tt
    expectedWalletAddressCallback}. It's a minor issue, as only the
  second call will finally set {\tt driven} with the value associated
  with the second {\tt drivenRoot}, as {\tt drivenRoot} is correctly
  checked in {\tt expectedWalletAddressCallback}. The only drawback is
  a potential small loss in gas. Anyway, this would not be possible if
  {\tt drivenRoot} was a static global variable of the contract.
\item Minor Issue: It would probably be better to use {\tt
  require(!statusConnected,..)} instead of {\tt
  if(!statusConnected){..}} to fail in case of called twice.
\item \issueError{}
\item \issueEncodeBody{}
\end{itemize}


\begin{lstlisting}[firstnumber=60]
  function deployEmptyWallet(address root) public override {
    require(msg.sender == dexclient, 101);
    require(!(msg.value < GRAMS_TO_ROOT * 2), 103);
    tvm.rawReserve(address(this).balance - msg.value, 2);
    if (!statusConnected) {
      drivenRoot = root;
      TvmCell bodyD = tvm.encodeBody(IRootTokenContract(root).deployEmptyWallet, GRAMS_TO_NEW_WALLET, 0, address(this), dexclient);
      root.transfer({value:GRAMS_TO_ROOT, bounce:true, body:bodyD});
      TvmCell bodyA = tvm.encodeBody(IRootTokenContract(root).sendExpectedWalletAddress, 0, address(this), address(this));
      root.transfer({value:GRAMS_TO_ROOT, bounce:true, body:bodyA});
      dexclient.transfer({value: 0, bounce:true, flag: 128});
    } else {
      dexclient.transfer({value: 0, bounce:true, flag: 128});
    }
  }
\end{lstlisting}

\subsection{Function expectedWalletAddressCallback}

\begin{itemize}
\item Minor Issue: this method should check {\tt require(!statusConnected,..)}
\item \issueError{}
\item \issueEncodeBody{}
\end{itemize}

\begin{lstlisting}[firstnumber=77]
  function expectedWalletAddressCallback(address wallet, uint256 wallet_public_key, address owner_address) public override {
    require(msg.sender == drivenRoot && wallet_public_key == 0 && owner_address == address(this), 102);
    tvm.rawReserve(address(this).balance - msg.value, 2);
    statusConnected = true;
    driven = wallet;
    TvmCell body = tvm.encodeBody(IDEXConnect(dexclient).connectCallback, wallet);
    dexclient.transfer({value: 0, bounce:true, flag: 128, body:body});
  }
\end{lstlisting}

\subsection{Function getBalance}

\begin{itemize}
\item Minor issue: is there a good reason to use {\tt
  checkOwnerAndAccept} to allow the user to spend gas to get the
  balance when this action can be performed without spending gas
  (through the GraphQL interface or through get-methods executed
  locally).
\end{itemize}

\begin{lstlisting}[firstnumber=124]
  function getBalance() public pure checkOwnerAndAccept returns (uint128 balance){
    balance = address(this).balance;
  }
\end{lstlisting}

\subsection{Function setBouncedCallback}

\begin{itemize}
\item Minor Issue: there is no real need for {\tt setBouncedCallback}
  and {\tt setTransferCallback} to be in two different methods instead
  of having a single method performing both actions, as they are
  always called together.
\item Minor Issue: this method should check {\tt require(statusConnected,..)}
\item \issueError{}
\item \issueEncodeBody{}
\end{itemize}

\begin{lstlisting}[firstnumber=95]
  function setBouncedCallback() public override {
    require(msg.sender == dexclient, 101);
    tvm.rawReserve(address(this).balance - msg.value, 2);
    TvmCell body = tvm.encodeBody(ITONTokenWallet(driven).setBouncedCallback, dexclient);
    driven.transfer({value: 0, bounce:true, flag: 128, body:body});
  }
\end{lstlisting}

\subsection{Function setTransferCallback}

\begin{itemize}
\item Minor Issue: this method should check {\tt require(statusConnected,..)}
\item \issueError{}
\item \issueEncodeBody{}
\end{itemize}



\begin{lstlisting}[firstnumber=87]
  function setTransferCallback() public override {
    require(msg.sender == dexclient, 101);
    tvm.rawReserve(address(this).balance - msg.value, 2);
    TvmCell body = tvm.encodeBody(ITONTokenWallet(driven).setReceiveCallback, dexclient, true);
    driven.transfer({value: 0, bounce:true, flag: 128, body:body});
  }
\end{lstlisting}

\subsection{Function transfer}


\begin{itemize}
\item Minor Issue: this method should check {\tt require(statusConnected,..)}
\item \issueError{}
\item \issueEncodeBody{}
\end{itemize}



\begin{lstlisting}[firstnumber=108]
  function transfer(address to, uint128 tokens, TvmCell payload) public override {
    require(msg.sender == dexclient, 101);
    tvm.rawReserve(address(this).balance - msg.value, 2);
    TvmCell body = tvm.encodeBody(ITONTokenWallet(driven).transfer, to, tokens, 0, dexclient, true, payload);
    driven.transfer({value: 0, bounce:true, flag: 128, body:body});
  }
\end{lstlisting}

\section{Internal Method Definitions}


\subsection{Function getQuotient}

\begin{itemize}
\item Minor Issue: This function is unused.
\item \issueInternal
\end{itemize}

\begin{lstlisting}[firstnumber=48]
  function getQuotient(uint128 arg0, uint128 arg1, uint128 arg2) private inline pure returns (uint128) {
    (uint128 quotient, ) = math.muldivmod(arg0, arg1, arg2);
    return quotient;
  }
\end{lstlisting}

\subsection{Function getRemainder}

\begin{itemize}
\item Minor Issue: This function is unused.
\item \issueInternal
\end{itemize}

\begin{lstlisting}[firstnumber=54]
  function getRemainder(uint128 arg0, uint128 arg1, uint128 arg2) private inline pure returns (uint128) {
    (, uint128 remainder) = math.muldivmod(arg0, arg1, arg2);
    return remainder;
  }
\end{lstlisting}
