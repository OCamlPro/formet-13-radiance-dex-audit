
\chapter{Contract DEXConnector}


In file {\tt DEXConnector.sol}

\section{Contract Inheritance}


\noindent\begin{tabular}{|l|p{5cm}|}\hline
IExpectedWalletAddressCallback & \\\hline
IDEXConnector & \\\hline
\end{tabular}


\section{Constant Definitions}

\begin{itemize}
\item Minor issue: see Better Units for Big Numbers (\ref{readability:numbers})
\end{itemize}

\ifsoldraft
\noindent\begin{tabular}{|p{12cm}|}\hline
\rowcolor{green}Advises
\\\hline
Use a naming convention to distinguish constants from other, such as all uppercase names.
\\\hline
Use \verb+ton+ unit instead of nanotons for cost constants to avoid numbers with too many zeroes.
\\\hline\end{tabular}
\fi

\ifsoltables
\noindent\begin{tabular}{|l|l|p{5cm}|}\hline
uint128 & GRAMS\_{}TO\_{}ROOT & Initialized to 500000000  \\\hline
uint128 & GRAMS\_{}TO\_{}NEW\_{}WALLET & Initialized to 250000000  \\\hline
\end{tabular}
\fi


\begin{lstlisting}[firstnumber=19]
  uint128 constant GRAMS_TO_ROOT = 500000000;
\end{lstlisting}

\begin{lstlisting}[firstnumber=20]
  uint128 constant GRAMS_TO_NEW_WALLET = 250000000;
\end{lstlisting}

\section{Static Variable Definitions}


\begin{itemize}
\item Minor issue: see Typography of Static Variables (\ref{readability:static})
\item Minor issue: why is {\tt drivenRoot} not a static variable ? it
  looks like there is only one possible {\tt DEXConnector} for a given
  pair of {\tt DEXClient} and {\tt RootTokenContract}.
\end{itemize}

\begin{lstlisting}[firstnumber=15]
  uint256 static public soUINT;
\end{lstlisting}

\begin{lstlisting}[firstnumber=16]
  address static public dexclient;
\end{lstlisting}

\section{Variable Definitions}

\begin{itemize}
\item Minor issue: see Typography of Global Variables (\ref{readability:global})
\end{itemize}

\begin{lstlisting}[firstnumber=22]
  address public drivenRoot;
\end{lstlisting}

\begin{lstlisting}[firstnumber=23]
  address public driven;
\end{lstlisting}

\begin{lstlisting}[firstnumber=24]
  bool public statusConnected;
\end{lstlisting}

\section{Modifier Definitions}

\subsection{Modifier alwaysAccept}

\begin{itemize}
\item Critical issue: see Accept Methods withtout Checks (\ref{accept:all}).
  This modifier should be removed.
\end{itemize}

\begin{lstlisting}[firstnumber=27]
	modifier alwaysAccept {
		tvm.accept();
		_;
	}
\end{lstlisting}

\subsection{Modifier checkOwnerAndAccept}

\begin{itemize}
\item Minor issue: see Naming of Errors
  (\ref{readability:errors}). Constants should be used instead of
  direct numbers.
\end{itemize}

\begin{lstlisting}[firstnumber=32]
  modifier checkOwnerAndAccept {
    // Check that message from contract owner.
    require(msg.sender == dexclient, 101);
    tvm.accept();
    _;
  }
\end{lstlisting}

\section{Constructor Definitions}


\subsection{Constructor}

\begin{lstlisting}[firstnumber=39]
  constructor() public checkOwnerAndAccept {
      statusConnected = false;
  }
\end{lstlisting}

\section{Public Method Definitions}


\subsection{Receive function}

\begin{lstlisting}[firstnumber=129]
  receive() external {
  }
\end{lstlisting}

\subsection{Function burn}

\begin{lstlisting}[firstnumber=116]
  function burn(uint128 tokens, address callback_address, TvmCell callback_payload) public override {
    require(msg.sender == dexclient, 101);
    tvm.rawReserve(address(this).balance - msg.value, 2);
    TvmCell body = tvm.encodeBody(IBurnableByOwnerTokenWallet(driven).burnByOwner, tokens, 0, dexclient, callback_address, callback_payload);
    driven.transfer({value: 0, bounce:true, flag: 128, body:body});
  }
\end{lstlisting}

\subsection{Function deployEmptyWallet}


\ifsoltables
\noindent\begin{tabular}{|l|l|p{5cm}|}\hline
\multicolumn{3}{|l|}{\bf Parameters}\\\hline
\tt address & \tt root &\\\hline
\end{tabular}
\fi



\begin{lstlisting}[firstnumber=60]
  function deployEmptyWallet(address root) public override {
    require(msg.sender == dexclient, 101);
    require(!(msg.value < GRAMS_TO_ROOT * 2), 103);
    tvm.rawReserve(address(this).balance - msg.value, 2);
    if (!statusConnected) {
      drivenRoot = root;
      TvmCell bodyD = tvm.encodeBody(IRootTokenContract(root).deployEmptyWallet, GRAMS_TO_NEW_WALLET, 0, address(this), dexclient);
      root.transfer({value:GRAMS_TO_ROOT, bounce:true, body:bodyD});
      TvmCell bodyA = tvm.encodeBody(IRootTokenContract(root).sendExpectedWalletAddress, 0, address(this), address(this));
      root.transfer({value:GRAMS_TO_ROOT, bounce:true, body:bodyA});
      dexclient.transfer({value: 0, bounce:true, flag: 128});
    } else {
      dexclient.transfer({value: 0, bounce:true, flag: 128});
    }
  }
\end{lstlisting}

\subsection{Function expectedWalletAddressCallback}


\ifsoltables
\noindent\begin{tabular}{|l|l|p{5cm}|}\hline
\multicolumn{3}{|l|}{\bf Parameters}\\\hline
\tt address & \tt wallet &\\\hline
\tt uint256 & \tt wallet\_{}public\_{}key &\\\hline
\tt address & \tt owner\_{}address &\\\hline
\end{tabular}
\fi



\begin{lstlisting}[firstnumber=77]
  function expectedWalletAddressCallback(address wallet, uint256 wallet_public_key, address owner_address) public override {
    require(msg.sender == drivenRoot && wallet_public_key == 0 && owner_address == address(this), 102);
    tvm.rawReserve(address(this).balance - msg.value, 2);
    statusConnected = true;
    driven = wallet;
    TvmCell body = tvm.encodeBody(IDEXConnect(dexclient).connectCallback, wallet);
    dexclient.transfer({value: 0, bounce:true, flag: 128, body:body});
  }
\end{lstlisting}

\subsection{Function getBalance}


\ifsoltables
\noindent\begin{tabular}{|l|l|p{5cm}|}\hline
\multicolumn{3}{|l|}{\bf Returns}\\\hline
\tt uint128 & \tt balance &\\\hline
\multicolumn{3}{|l|}{\bf Modifiers}\\\hline
\tt checkOwnerAndAccept & {\em no args} &\\\hline
\end{tabular}
\fi



\begin{lstlisting}[firstnumber=124]
  function getBalance() public pure checkOwnerAndAccept returns (uint128 balance){
    balance = address(this).balance;
  }
\end{lstlisting}

\subsection{Function setBouncedCallback}



\begin{lstlisting}[firstnumber=95]
  function setBouncedCallback() public override {
    require(msg.sender == dexclient, 101);
    tvm.rawReserve(address(this).balance - msg.value, 2);
    TvmCell body = tvm.encodeBody(ITONTokenWallet(driven).setBouncedCallback, dexclient);
    driven.transfer({value: 0, bounce:true, flag: 128, body:body});
  }
\end{lstlisting}

\subsection{Function setTransferCallback}



\begin{lstlisting}[firstnumber=87]
  function setTransferCallback() public override {
    require(msg.sender == dexclient, 101);
    tvm.rawReserve(address(this).balance - msg.value, 2);
    TvmCell body = tvm.encodeBody(ITONTokenWallet(driven).setReceiveCallback, dexclient, true);
    driven.transfer({value: 0, bounce:true, flag: 128, body:body});
  }
\end{lstlisting}

\subsection{Function transfer}


\ifsoltables
\noindent\begin{tabular}{|l|l|p{5cm}|}\hline
\multicolumn{3}{|l|}{\bf Parameters}\\\hline
\tt address & \tt to &\\\hline
\tt uint128 & \tt tokens &\\\hline
\tt TvmCell & \tt payload &\\\hline
\end{tabular}
\fi



\begin{lstlisting}[firstnumber=108]
  function transfer(address to, uint128 tokens, TvmCell payload) public override {
    require(msg.sender == dexclient, 101);
    tvm.rawReserve(address(this).balance - msg.value, 2);
    TvmCell body = tvm.encodeBody(ITONTokenWallet(driven).transfer, to, tokens, 0, dexclient, true, payload);
    driven.transfer({value: 0, bounce:true, flag: 128, body:body});
  }
\end{lstlisting}

\section{Internal Method Definitions}


\subsection{Function getQuotient}


\ifsoltables
\noindent\begin{tabular}{|l|l|p{5cm}|}\hline
\multicolumn{3}{|l|}{\bf Parameters}\\\hline
\tt uint128 & \tt arg0 &\\\hline
\tt uint128 & \tt arg1 &\\\hline
\tt uint128 & \tt arg2 &\\\hline
\multicolumn{3}{|l|}{\bf Returns}\\\hline
\tt uint128 & {\em no name} &\\\hline
\end{tabular}
\fi



\begin{lstlisting}[firstnumber=48]
  function getQuotient(uint128 arg0, uint128 arg1, uint128 arg2) private inline pure returns (uint128) {
    (uint128 quotient, ) = math.muldivmod(arg0, arg1, arg2);
    return quotient;
  }
\end{lstlisting}

\subsection{Function getRemainder}


\ifsoltables
\noindent\begin{tabular}{|l|l|p{5cm}|}\hline
\multicolumn{3}{|l|}{\bf Parameters}\\\hline
\tt uint128 & \tt arg0 &\\\hline
\tt uint128 & \tt arg1 &\\\hline
\tt uint128 & \tt arg2 &\\\hline
\multicolumn{3}{|l|}{\bf Returns}\\\hline
\tt uint128 & {\em no name} &\\\hline
\end{tabular}
\fi


\begin{lstlisting}[firstnumber=54]
  function getRemainder(uint128 arg0, uint128 arg1, uint128 arg2) private inline pure returns (uint128) {
    (, uint128 remainder) = math.muldivmod(arg0, arg1, arg2);
    return remainder;
  }
\end{lstlisting}
