
\section{Module "DEXConnector.sol"}


\subsection{Pragmas}


\noindent\begin{tabular}{|l|l|p{5cm}|}\hline
ton & -solidity \^{}0.45.0 &\\\hline
AbiHeader &  expire &\\\hline
AbiHeader &  pubkey &\\\hline
AbiHeader &  time &\\\hline
\end{tabular}


\subsection{Imports}


\noindent\begin{tabular}{|l|l|p{5cm}|}\hline
./interfaces/IRootTokenContract.sol &\\\hline
./interfaces/ITONTokenWallet.sol &\\\hline
./interfaces/IExpectedWalletAddressCallback.sol &\\\hline
./interfaces/IBurnableByOwnerTokenWallet.sol &\\\hline
./interfaces/IDEXConnector.sol &\\\hline
./interfaces/IDEXConnect.sol &\\\hline
\end{tabular}


\subsection{Contract DEXConnector}


In file {\tt DEXConnector.sol}

\subsubsection{Contract Inheritance}


\noindent\begin{tabular}{|l|p{5cm}|}\hline
IExpectedWalletAddressCallback & \\\hline
IDEXConnector & \\\hline
\end{tabular}


\subsubsection{Constant Definitions}


\ifsoldraft
\noindent\begin{tabular}{|p{12cm}|}\hline
\rowcolor{green}Advises
\\\hline
Use a naming convention to distinguish constants from other, such as all uppercase names.
\\\hline
Use \verb+ton+ unit instead of nanotons for cost constants to avoid numbers with too many zeroes.
\\\hline\end{tabular}
\fi

\ifsoltables
\noindent\begin{tabular}{|l|l|p{5cm}|}\hline
uint128 & GRAMS\_{}TO\_{}ROOT & Initialized to 500000000  \\\hline
uint128 & GRAMS\_{}TO\_{}NEW\_{}WALLET & Initialized to 250000000  \\\hline
\end{tabular}
\fi


\begin{lstlisting}[firstnumber=19]
  uint128 constant GRAMS_TO_ROOT = 500000000;
\end{lstlisting}

\begin{lstlisting}[firstnumber=20]
  uint128 constant GRAMS_TO_NEW_WALLET = 250000000;
\end{lstlisting}

\subsubsection{Static Variable Definitions}


\ifsoldraft
\noindent\begin{tabular}{|p{12cm}|}\hline
\rowcolor{green}Advises
\\\hline
Use a naming convention to distinguish static variables from global variables, such as \verb+s_+ prefix.
\\\hline\end{tabular}
\fi

\ifsoltables
\noindent\begin{tabular}{|l|l|p{5cm}|}\hline
uint256 & soUINT &  \\\hline
address & dexclient &  \\\hline
 & & used in @2.DEXConnector.transfer\\\hline
 & & used in @2.DEXConnector.transfer\\\hline
 & & used in @2.DEXConnector.setTransferCallback\\\hline
 & & used in @2.DEXConnector.setTransferCallback\\\hline
 & & used in @2.DEXConnector.setBouncedCallback\\\hline
 & & used in @2.DEXConnector.setBouncedCallback\\\hline
 & & used in @2.DEXConnector.expectedWalletAddressCallback\\\hline
 & & used in @2.DEXConnector.expectedWalletAddressCallback\\\hline
 & & used in @2.DEXConnector.deployEmptyWallet\\\hline
 & & used in @2.DEXConnector.deployEmptyWallet\\\hline
 & & used in @2.DEXConnector.deployEmptyWallet\\\hline
 & & used in @2.DEXConnector.deployEmptyWallet\\\hline
 & & used in @2.DEXConnector.burn\\\hline
 & & used in @2.DEXConnector.burn\\\hline
\end{tabular}
\fi


\begin{lstlisting}[firstnumber=15]
  uint256 static public soUINT;
\end{lstlisting}

\begin{lstlisting}[firstnumber=16]
  address static public dexclient;
\end{lstlisting}

\subsubsection{Variable Definitions}


\ifsoldraft
\noindent\begin{tabular}{|p{12cm}|}\hline
\rowcolor{green}Advises
\\\hline
Use a naming convention to distinguish global variables from local variables, such as \verb+g_+ or \verb+m_+ prefix.
\\\hline\end{tabular}
\fi

\ifsoltables
\noindent\begin{tabular}{|l|l|p{5cm}|}\hline
address & drivenRoot &  \\\hline
 & & used in @2.DEXConnector.expectedWalletAddressCallback\\\hline
 & & assigned in @2.DEXConnector.deployEmptyWallet\\\hline
 & & used in @2.DEXConnector.deployEmptyWallet\\\hline
address & driven &  \\\hline
 & & used in @2.DEXConnector.transfer\\\hline
 & & used in @2.DEXConnector.transfer\\\hline
 & & used in @2.DEXConnector.setTransferCallback\\\hline
 & & used in @2.DEXConnector.setTransferCallback\\\hline
 & & used in @2.DEXConnector.setBouncedCallback\\\hline
 & & used in @2.DEXConnector.setBouncedCallback\\\hline
 & & assigned in @2.DEXConnector.expectedWalletAddressCallback\\\hline
 & & used in @2.DEXConnector.expectedWalletAddressCallback\\\hline
 & & used in @2.DEXConnector.burn\\\hline
 & & used in @2.DEXConnector.burn\\\hline
bool & statusConnected &  \\\hline
 & & assigned in @2.DEXConnector.expectedWalletAddressCallback\\\hline
 & & used in @2.DEXConnector.expectedWalletAddressCallback\\\hline
 & & used in @2.DEXConnector.deployEmptyWallet\\\hline
 & & assigned in @2.DEXConnector.:constructor\\\hline
 & & used in @2.DEXConnector.:constructor\\\hline
\end{tabular}
\fi


\begin{lstlisting}[firstnumber=22]
  address public drivenRoot;
\end{lstlisting}

\begin{lstlisting}[firstnumber=23]
  address public driven;
\end{lstlisting}

\begin{lstlisting}[firstnumber=24]
  bool public statusConnected;
\end{lstlisting}

\subsubsection{Modifier Definitions}


\ifsoldraft
\noindent\begin{tabular}{|p{12cm}|}\hline
\rowcolor{green}Advises
\\\hline
Calling tvm.accept() without checking pubkey should not be allowed
\\\hline\end{tabular}
\fi

\paragraph{Modifier alwaysAccept}


\begin{lstlisting}[firstnumber=27]
	modifier alwaysAccept {
		tvm.accept();
		_;
	}
\end{lstlisting}

\paragraph{Modifier checkOwnerAndAccept}


\begin{lstlisting}[firstnumber=32]
  modifier checkOwnerAndAccept {
    // Check that message from contract owner.
    require(msg.sender == dexclient, 101);
    tvm.accept();
    _;
  }
\end{lstlisting}

\subsubsection{Constructor Definitions}


\paragraph{Constructor}

\issueCritical{Constructor for DEXConnector}{loren ipsum  loren ipsum  loren ipsum loren ipsum loren ipsum loren ipsum loren ipsum loren ipsum loren ipsum loren ipsum loren ipsum loren ipsum loren ipsum loren ipsum loren ipsum loren ipsum loren ipsum loren ipsum

loren ipsum loren ipsum loren ipsum loren ipsum loren ipsum loren ipsum
loren ipsum loren ipsum loren ipsum }

\ifsoldraft
\noindent\begin{tabular}{|p{12cm}|}\hline
\rowcolor{green}Advises
\\\hline
Check who can call the constructor. If the constructor sets global values, only legitimate users should be allowed.
\\\hline
Check that every argument is protected by a require().
\\\hline
If external users are allowed, their pubkey should be verified (\verb+require(msg.pubkey() != 0 && msg.pubkey() == tvm.pubkey(),100)+ , and tvm.accept() should be called.
\\\hline\end{tabular}
\fi

\ifsoltables
\noindent\begin{tabular}{|l|l|p{5cm}|}\hline
\multicolumn{3}{|l|}{\bf Modifiers}\\\hline
\tt checkOwnerAndAccept & {\em no args} &\\\hline
\end{tabular}
\fi

\vspace{2cm}

\begin{lstlisting}[firstnumber=39]
  constructor() public checkOwnerAndAccept {
      statusConnected = false;
  }
\end{lstlisting}

\subsubsection{Public Method Definitions}


\paragraph{Receive function}

\vspace{2cm}

\begin{lstlisting}[firstnumber=129]
  receive() external {
  }
\end{lstlisting}

\paragraph{Function burn}


\ifsoltables
\noindent\begin{tabular}{|l|l|p{5cm}|}\hline
\multicolumn{3}{|l|}{\bf Parameters}\\\hline
\tt uint128 & \tt tokens &\\\hline
\tt address & \tt callback\_{}address &\\\hline
\tt TvmCell & \tt callback\_{}payload &\\\hline
\end{tabular}
\fi

\vspace{2cm}

\begin{lstlisting}[firstnumber=116]
  function burn(uint128 tokens, address callback_address, TvmCell callback_payload) public override {
    require(msg.sender == dexclient, 101);
    tvm.rawReserve(address(this).balance - msg.value, 2);
    TvmCell body = tvm.encodeBody(IBurnableByOwnerTokenWallet(driven).burnByOwner, tokens, 0, dexclient, callback_address, callback_payload);
    driven.transfer({value: 0, bounce:true, flag: 128, body:body});
  }
\end{lstlisting}

\paragraph{Function deployEmptyWallet}


\ifsoltables
\noindent\begin{tabular}{|l|l|p{5cm}|}\hline
\multicolumn{3}{|l|}{\bf Parameters}\\\hline
\tt address & \tt root &\\\hline
\end{tabular}
\fi

\vspace{2cm}

\begin{lstlisting}[firstnumber=60]
  function deployEmptyWallet(address root) public override {
    require(msg.sender == dexclient, 101);
    require(!(msg.value < GRAMS_TO_ROOT * 2), 103);
    tvm.rawReserve(address(this).balance - msg.value, 2);
    if (!statusConnected) {
      drivenRoot = root;
      TvmCell bodyD = tvm.encodeBody(IRootTokenContract(root).deployEmptyWallet, GRAMS_TO_NEW_WALLET, 0, address(this), dexclient);
      root.transfer({value:GRAMS_TO_ROOT, bounce:true, body:bodyD});
      TvmCell bodyA = tvm.encodeBody(IRootTokenContract(root).sendExpectedWalletAddress, 0, address(this), address(this));
      root.transfer({value:GRAMS_TO_ROOT, bounce:true, body:bodyA});
      dexclient.transfer({value: 0, bounce:true, flag: 128});
    } else {
      dexclient.transfer({value: 0, bounce:true, flag: 128});
    }
  }
\end{lstlisting}

\paragraph{Function expectedWalletAddressCallback}


\ifsoltables
\noindent\begin{tabular}{|l|l|p{5cm}|}\hline
\multicolumn{3}{|l|}{\bf Parameters}\\\hline
\tt address & \tt wallet &\\\hline
\tt uint256 & \tt wallet\_{}public\_{}key &\\\hline
\tt address & \tt owner\_{}address &\\\hline
\end{tabular}
\fi

\vspace{2cm}

\begin{lstlisting}[firstnumber=77]
  function expectedWalletAddressCallback(address wallet, uint256 wallet_public_key, address owner_address) public override {
    require(msg.sender == drivenRoot && wallet_public_key == 0 && owner_address == address(this), 102);
    tvm.rawReserve(address(this).balance - msg.value, 2);
    statusConnected = true;
    driven = wallet;
    TvmCell body = tvm.encodeBody(IDEXConnect(dexclient).connectCallback, wallet);
    dexclient.transfer({value: 0, bounce:true, flag: 128, body:body});
  }
\end{lstlisting}

\paragraph{Function getBalance}


\ifsoltables
\noindent\begin{tabular}{|l|l|p{5cm}|}\hline
\multicolumn{3}{|l|}{\bf Returns}\\\hline
\tt uint128 & \tt balance &\\\hline
\multicolumn{3}{|l|}{\bf Modifiers}\\\hline
\tt checkOwnerAndAccept & {\em no args} &\\\hline
\end{tabular}
\fi

\vspace{2cm}

\begin{lstlisting}[firstnumber=124]
  function getBalance() public pure checkOwnerAndAccept returns (uint128 balance){
    balance = address(this).balance;
  }
\end{lstlisting}

\paragraph{Function setBouncedCallback}

\vspace{2cm}

\begin{lstlisting}[firstnumber=95]
  function setBouncedCallback() public override {
    require(msg.sender == dexclient, 101);
    tvm.rawReserve(address(this).balance - msg.value, 2);
    TvmCell body = tvm.encodeBody(ITONTokenWallet(driven).setBouncedCallback, dexclient);
    driven.transfer({value: 0, bounce:true, flag: 128, body:body});
  }
\end{lstlisting}

\paragraph{Function setTransferCallback}

\vspace{2cm}

\begin{lstlisting}[firstnumber=87]
  function setTransferCallback() public override {
    require(msg.sender == dexclient, 101);
    tvm.rawReserve(address(this).balance - msg.value, 2);
    TvmCell body = tvm.encodeBody(ITONTokenWallet(driven).setReceiveCallback, dexclient, true);
    driven.transfer({value: 0, bounce:true, flag: 128, body:body});
  }
\end{lstlisting}

\paragraph{Function transfer}


\ifsoltables
\noindent\begin{tabular}{|l|l|p{5cm}|}\hline
\multicolumn{3}{|l|}{\bf Parameters}\\\hline
\tt address & \tt to &\\\hline
\tt uint128 & \tt tokens &\\\hline
\tt TvmCell & \tt payload &\\\hline
\end{tabular}
\fi

\vspace{2cm}

\begin{lstlisting}[firstnumber=108]
  function transfer(address to, uint128 tokens, TvmCell payload) public override {
    require(msg.sender == dexclient, 101);
    tvm.rawReserve(address(this).balance - msg.value, 2);
    TvmCell body = tvm.encodeBody(ITONTokenWallet(driven).transfer, to, tokens, 0, dexclient, true, payload);
    driven.transfer({value: 0, bounce:true, flag: 128, body:body});
  }
\end{lstlisting}

\subsubsection{Internal Method Definitions}


\paragraph{Function getQuotient}


\ifsoltables
\noindent\begin{tabular}{|l|l|p{5cm}|}\hline
\multicolumn{3}{|l|}{\bf Parameters}\\\hline
\tt uint128 & \tt arg0 &\\\hline
\tt uint128 & \tt arg1 &\\\hline
\tt uint128 & \tt arg2 &\\\hline
\multicolumn{3}{|l|}{\bf Returns}\\\hline
\tt uint128 & {\em no name} &\\\hline
\end{tabular}
\fi

\vspace{2cm}

\begin{lstlisting}[firstnumber=48]
  function getQuotient(uint128 arg0, uint128 arg1, uint128 arg2) private inline pure returns (uint128) {
    (uint128 quotient, ) = math.muldivmod(arg0, arg1, arg2);
    return quotient;
  }
\end{lstlisting}

\paragraph{Function getRemainder}


\ifsoltables
\noindent\begin{tabular}{|l|l|p{5cm}|}\hline
\multicolumn{3}{|l|}{\bf Parameters}\\\hline
\tt uint128 & \tt arg0 &\\\hline
\tt uint128 & \tt arg1 &\\\hline
\tt uint128 & \tt arg2 &\\\hline
\multicolumn{3}{|l|}{\bf Returns}\\\hline
\tt uint128 & {\em no name} &\\\hline
\end{tabular}
\fi

\vspace{2cm}

\begin{lstlisting}[firstnumber=54]
  function getRemainder(uint128 arg0, uint128 arg1, uint128 arg2) private inline pure returns (uint128) {
    (, uint128 remainder) = math.muldivmod(arg0, arg1, arg2);
    return remainder;
  }
\end{lstlisting}
