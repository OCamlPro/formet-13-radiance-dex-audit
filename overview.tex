
The infrastructure is composed of a set of DEX specific contracts,
associated with tokens contracts (developed by Broxus, to the best of
our knownledge).

The DEX contracts are :

\begin{description}
\item[DEXRoot:] The ``root'' contract, used to perform global
  operations, such as creating ``client'' contracts;
\item[DEXClient:] The contract with which a user may interact with the
  system.
\item[DEXPair:] The contract associated with a given pair of tokens
  (root token contracts)
\item[DEXConnector:] A simplified interface to interact with token
  contracts. The goal is probably to be able to interact with
  different implementations/interfaces of token contracts.
\end{description}

The token contracts are :

\begin{description}
\item[RootTokenContract:] The root token contract, shared by all the
  wallet contracts for a given token;
\item[TONTokenWallet:] The wallet contract, containing the balance
  associated either with a public key or (exclusive) a contract
  address;
\end{description}

Compared to
\url{https://github.com/broxus/broxus/ton-eth-bridge-token-contracts/},
the two token contracts have only been modified to change the {\tt
  ton-solidity} pragma version.

All the DEX contracts use a static {\tt soUINT} field to be able to
instanciate several ones for a given public key or other static field.
